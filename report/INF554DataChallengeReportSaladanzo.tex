\documentclass[twocolumn]{article}
\usepackage{amsmath,empheq,hyperref}
\usepackage[justification=centering]{caption}
\usepackage[column=O]{cellspace}
\hypersetup{colorlinks=true, urlcolor=cyan}

\title{Subevent Detection in Soccer Matches from Tweets}
\author{Saleh Shamloo Ahmadi, Adama Koïta, Enzo Gelas}
\date{December 10, 2024}

\begin{document}
\maketitle
\section{Data Processing and Feature Extraction}
\subsection{Models trained on Embedding}
Our data processing pipline is as follows:
\begin{enumerate}
    \item \textbf{Remove duplicate tweets (retweets).} Examining the data set, mostly the
        promotional tweets were retweeted a lot, and there is no benefit in repeating the
        information in relevant retweets.
    \item \textbf{Remove all characters except for the hashtag character, the ampersand, the dollar
        sign, and emojis.} We did not remove all characters, since some of them convey meaning. For
        the GloVe embedding, since the emoji embeddings were sparse, we just removed every
        punctuation.
    \item \text{Lowercase letters.} Cased analysis in this case is not important, as we are mostly
        dealing with sentiments and not semantics.
    \item \textbf{Do \emph{not} remove any stopwords.} This always reduced accuracy.
    \item \textbf{Generate word embeddings.} We use both our own custom FastText embedding and a
        GloVe embedding trained on tweets. We only use a 100-dimensional embedding, as more
        dimensions do not improve the accuracy.
\end{enumerate}

We used two different word embeddings:
\begin{enumerate}
    \item \textbf{An unsupervised FastText\cite{bojanowski2016enriching} model trained by
        ourselves.} We used all the available text of the tweets from the dataset for this step.
        A more rigorous model might be trained on tweets from more tweets related to football
        matches. Using the FastText model has the advantage of not needing any lemmatization or
        splitting of words, since it uses ngrams to extract subword-level meaning.
    \item \textbf{A GloVe\cite{pennington2014glove} model trained on tweets.} Since we again did
        not do any lemmatization, since this decreased the accuracy of our model significantly.
\end{enumerate}

We extracted these features:
\begin{enumerate}
    \item \textbf{Mean embedding vector (normalized by FastText) of the tweets of each time period,
        minus the mean embedding vector of the match.} Removing the background noise of each match
        was crucial to achieving higher accuracy, and gave the biggest accuracy boost to our model.
    \item \textbf{Standard deviation of the embedding vectors of each time period, minus the mean
        standard deviation of the match's embedding vectors.} This helps indicate when a time period
        has more noisy data and there is less confidence about the meaning of the embedding vector.
    \item \textbf{The mean embedding vectors of the previous and next time period's tweets.} This
        helps add some information about the time evolution of the tweets.
    \item \textbf{The ratio of the number of tweets in each time period to the number of tweets in
        the match.} This is a good-enough feature for training a model on its own, and it helps
        with the accuracy of our model too.
\end{enumerate}

Another very important step in processing the data is to split into different matches for
cross-validation, not individual tweets. Mixing tweets from different matches causes leakage from
the validation set into the training set, overfitting the models.

\subsection{LLM Tweet Classifiers}
For training Large Language Models (LLMs) on the dataset, we chose to classify each tweet seperately
and later predict the subevent classification from the confidence of tweet classification of each
model. We also tried concatenating all words of each time period and train a model on the embedding
obtained from that, but this proved to be ineffective. We also tried training our own attention
model, but this proved even more ineffective.

\subsection{Word Frequencies Model}
For training a more simplistic word frequency model, we extracted the frequency of these words in
the dataset: \texttt{[`win', `lose', `draw', `goal', `red', `yellow', `penalty', `foul', `offfside',
`corner', `free', `kick', `score', `assist', `pass', `tackle']}. These were chosen after a word
frequency analysis on the dataset. We chose from the most frequent words, the ones that seemed to
be more relevant to the task. For this model, we did apply a lemmtizer to find verbs and plurals
too.

\section{Model Choice, Tuning, and Comparison}
\subsection{Model Choice}
For the word embeddings, doing preliminary analysis (i.e. no hyperparameter tuning), it was apparent
that tree boosting models produced the best results. Among these models, we decided to use XGBoost
\cite{chen2016xgboost} as it is the most flexible of these models and that the preliminary analysis
showed tree models achieve similar accuracies before tuning.

For the LLM tweet classifier, we chose DistilBERT \cite{sanh2020distilbert} as training the full
BERT model proved too resource-intensive. After producing the classification confidence output, we
trained an XGBoost model on the output.

We also trained a graph-of-words model inspired by Meladianos \textit{et al.} (2018)
\cite{meladianos2018subevent}. However, we could not produce a model with the same accuracy.

For the word frequencies, we trained two decision tree classifiers based on these criteria:
\begin{enumerate}
    \item Number of tweets with the words in them divided by the number of all tweets.
    \item Number of word occurances divided by the number of tweets.
\end{enumerate}
To improve the performance further, we also normalized by the word frequencies per match.
This model did not show future potential, so we abandoned it before using a more complex classifier
to train it. We left it as a baseline.

\subsection{Tuning}
For the XGBoost model, we tuned the hyperparameters in batches that are affected by each other:
\begin{enumerate}
    \item \texttt{max\_depth} and \texttt{min\_child\_weight},
    \item \texttt{subsample} and \texttt{colsample\_bytree},
    \item \texttt{learning\_rate} and \texttt{n\_estimators}.
\end{enumerate}
This improved the accuracy by 2\%.

\subsection{Results and Comparison}
The model accuracies are compared in Table \ref{tab:acc}. As you can see, the XGBoost model trained
on the FastText embeddings worked best. However, it only translated the performance from the
cross-validation to the never-before-seen test set when we subtracted the mean word embedding of the
matches. The LLM model completely overfit the classification training data (instead of sentiment
analysis, it seems to have memorized the dataset). The word frequency model performed poorly, as it
was too simplistic. The graph-of-words model did not perform well either, which may be due to
implementation errors.

\begin{table*}[htb!]
    \centering
    \caption{Accuracy comparison for different models.}
    \label{tab:acc}
    \begin{tabular}{|c|c|c|}
        \hline
        Model & Cross-validation accuracy & Test accuracy \\ \hline
        XGBoost on FastText, Match Mean Subtracted & 0.76472 & 0.73046 \\ \hline
        XGBoost on FastText, No Subtraction & 0.73351 & 0.68750 \\ \hline
        XGBoost on GloVe, Match Mean Subtracted & 0.73512 & \textbf{0.74218} \\ \hline
        DistilBERT & 0.90189 & 0.60156 \\ \hline
        Word frequencies, word-level & 0.66213 & 0.62109 \\ \hline
        Word frequencies, tweet-level & 0.61722 & 0.56640 \\ \hline
        Graph-of-words & 0.65342 & 0.62890 \\ \hline
    \end{tabular}
\end{table*}

We conclude that in this task, the most important step was to remove match biases. Other steps, such
as including tweet frequency or the embeddings of the previous and next time periods did not help as
much. The LLM could not be trained in a way to avoid overfitting, as it is not well suited for this
type of task (``custom'' sentiments instead of real-life sentiments).
\bibliographystyle{plain}
\bibliography{ref}
\end{document}